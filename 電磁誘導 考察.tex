\documentclass[11pt]{jsarticle}
\usepackage{ascmac}
\begin{document}
\newpage
%------ここまで表紙------%
理工学基礎実験 1A「電磁誘導」 	学籍番号 1X18D078-4 氏名 吉田 陽向 \\
結果・考察 まとめシート \\
(物理試問の日の出欠確認時に以下の課題をやっていない場合は,試問を受講できない)\\
\section{実験1 磁石がつくる磁束密度の測定} 
棒磁石のN極の先端からの距離$z$と,その位置における磁束密度$B_z$の関係 
図1の両対数グラフを見ると,測定データは領域によっては中央に引かれた点線と同程度の傾きになっている.
このことから,磁束密度$B_z$が距離$z$に対してどのような関係にあると判断できるか答えよ.(2)
\begin{screen}
    $B_z = \frac{C}{z^2} (Cは0以上の定数)$
\end{screen}
磁気に関するクーロンの法則を考えると,磁束密度$B_z$はN極の先端からの位置に対してテキストの式(2)のようになることが理論的に示される.
図1には,3つの磁極の強さ$m$を式(2)に代入して計算した理論曲線も引かれている.
実測したデータは,式(2)の理論曲線と一致していたか,最もよく一致する$m$の値はいくつであったか,を答えよ.
\begin{screen}
    m = 55μWbのとき、実測データに最も一致する理論曲線が得られた。
    しかし、zが大きくなると実測データが理論曲線から外れた(特にz=0.200の時)
\end{screen}
図1において,実測したデータや理論曲線の傾きは,領域によっては中央に引かれた点線の直線と一致するが領域によって外れているように見える.
それはなぜかを考察せよ.
\begin{screen}
    \
    \
    \
    \
\end{screen}
\section{実験2 磁石を上下させたときの波形の観測}
棒磁石をコイルに通過させたときの誘導起電力の測定を行った.
誘導起電力$V$とコイル内の磁束$\Phi$,通過速度$v$にはどのような関係があるか式で表し,法則名を答えよ.(2)
\begin{screen}
    $V = -N\frac{d\Phi}{dt} = -Nv\frac{d\Phi}{dx}(Nはコイルの巻き数)$\\
    \hspace{100pt} -----ファラデーの電磁誘導の法則
\end{screen}
棒磁石をコイルに通さないときの誘導起電力は\fbox{0}であった. \\
(i) N極を下向きにして棒磁石を下した場合には,最初に\fbox{負の方向}の起電力が生じた後に\fbox{正の方向}の起電力が発生した.
なぜこのようになったか考察せよ.(6)
\begin{screen}
    下向きの磁界を正とする。
    始め棒磁石がコイルよりも上部にある時は、磁石が運動するにつれ\
    \
    \
    \
    \
    \
\end{screen}
(ii) S極を下向きにした場合には,最初に\fbox{\phantom{正の方向}}の,後に\fbox{\phantom{負の方向}}の起電力が発生した.
(iii) N極を下向きにして通過速度を速くしたときの起電力の時間変化は,ゆっくり通過させたときに比べてどのような違いがあったか述べよ.
\begin{screen}

\end{screen}
\section{実験3 棒磁石の自由落下によりコイルに発生する起電力の測定}
①アクリルパイプに巻き付けたコイルの中にN極を下にした棒磁石を落下させたとき,起電力の極小値に比べて極大値の絶対値は\fbox{大きく}なった.
また,コイルの位置を変えて落下距離を長くするにつれて極小値(もしくは極大値)の絶対値は\fbox{大きく}なった.
これは上述のように,起電力$V$が$V=$\fbox{$v(t)\frac{d\Phi}{dh}$}という関係に従うため,
\fbox{重力}によって落下速度$v$が\fbox{増加}することによって起電力の大きさも\fbox{\phantom{ああああああああ}}からである.
一方,負の起電力を時間積分した値はコイル位置$h_0$に対して\fbox{依存しない}.
これは,起電力$V=$\fbox{$-\frac{d\Phi}{dt}$}を積分して$\int V dt=$\fbox{$-\Phi$}となるのでその絶対値は\fbox{コイルの磁束}と一致する.
また,\fbox{$\int_{t_1}^{t_0} V(t) dt = -\frac{ml}{2} \frac{1}{\sqrt{a^2+l^2/2}}$}(17)なので,起電力の積分値がコイル位置$h_0$に対して\fbox{無関係}と解釈できる.
②棒磁石の長さを$l$とし,落下前のN極の先端の位置とコイル中心の距離$h_0$と,
磁石の中心がコイルの中心を通過するときの時間$t_0$との関係は$h_0=$\fbox{$\frac{g}{2} *t^{2}-\frac{l}{2}$}と表せる.
つまり図4の傾きは\fbox{$\frac{g}{2}$},切片は\fbox{$-\frac{l}{2}$}を表していることになる.
最小二乗法により,傾きは\fbox{4.77},切片は\fbox{$-4.39*10^{-2}$}と求まった.
ゆえに重力加速度$g$は\fbox{9.54 $[m/s^{2}]$}と見積もられた.これらの値を文献値と比較し,考察せよ.
\begin{screen}
    \
    \
    \
    \
    \
    \
    \
    \
\end{screen}
\section{実験4 金属パイプ中での磁石の落下運動の観測}
① 銅パイプと真鍮パイプに巻き付けたコイルの中にネオジウム磁石を落下さたとき,実験3と同様の測定を行った.
図5から,$h_0$と$t_0$とは互いに\fbox{比例}関数の関係にあることが分かる.
つまり,速度$v$は\fbox{一定}になっている.
これは,テキスト式(15)から,速度$v$が$v=$\fbox{$\frac{Mg}{\Gamma}$}と表されることと定性的に一致する.
それぞれの直線の傾きと実験3で求めた$g$の値から,$\frac{M}{\Gamma}$を見積もったところ,
銅パイプの時は$\frac{M}{\Gamma} =$\fbox{$1.62*10^{-2}$},真鍮パイプの時は$\frac{M}{\Gamma} =$\fbox{$4.92*10^{-2}$}となった.
本実験において測定した時間$t$は$\frac{M}{\Gamma}$よりも十分に\fbox{大きい}ことから,磁石が\fbox{等速}運動していたと考えられる.
銅は真鍮に比べて電気抵抗率が小さい材質である.
式(15)と式(16)を見比べると抵抗$R$が小さい方が$\Gamma$が\fbox{大きく}なることが分かるので,
銅パイプの方が$\frac{M}{\Gamma}$が\fbox{小さく}なったと解釈できる.
②磁石が金属パイプ中を自由落下するときに抵抗力を受ける原理を説明せよ.
\begin{screen}
    \
    \
    \
    \
    \
    \
    \
    \
    \
\end{screen}
\end{document}